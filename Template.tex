\documentclass[11pt,spanish]{exam}
%Comentar la siguiente linea para esconder Soluciones
\printanswers

%codigo del curso en minusculas (fmm001)
%campus con primeras 4 letras (repu,caso,vina,conc,avar,leon,bell)
\usepackage[fms171,repu,borrador]{mat-unab}

% Los siguientes comandos pueden ser alterados de acuerdo a cada evaluacion.
\newcommand{\evaluacion}{Solemne 1}
%\newcommand{\evaluacion}{Solemne 1 - Forma A}
\newcommand{\fecha}{1 de septiembre de 2017}
\newcommand{\duracion}{80} % en minutos


\begin{document}

%% Genera el titulo de la prueba, tiempo, y consultas
\thetitle 

%% Activar \small en caso de que se exceda una pagina
%\small

\begin{questions}

%%% Cada problema se inicia con "\question". Es un equivalente a \item del ambiente "enumerate"
\question
La glicemia es la medida de concentración de glucosa libre en la sangre de las personas y se sabe que, históricamente, se distribuye normal con media 90 mg/dL y varianza 169 (mg/dL)$^2$. Se tomó una muestra aleatoria de pacientes de un laboratorio, donde se obtuvo lo siguiente:
\begin{center}
\begin{tabular}{cc}
\toprule
Glicemia (mg/dL) & N\no\ de Pacientes \\
\midrule
68 - 78 & 6 \\ 
78 - 88 & 13 \\
88 - 98 & 18 \\
98 - 108 & 11 \\
108 - 118 & 8 \\
\bottomrule
\end{tabular}
\end{center}

%%% Un ambiente para preguntas con varias partes. Cada parte se inicia con el comando "\part[PTJE]" donde PTJE es el numero ENTERO de puntos asociados a esta parte. Se pueden asignar medios puntos usando "\half", por ejemplo, "1\half" genera 1,5 puntos Si es necesario hacer subpartes se puede usar el ambiente \begin{subparts} \subpart ... \end{subparts}. Ver la documentacion de la clase "exam" para mas detalles.
\begin{parts}
 	\part[10] Las autoridades del Ministerio de Salud tienen la sospecha que el valor promedio de glicemia ha aumentado, pero ha disminuido la variabilidad. ¿Qué puede indicar usted al respecto? Justifique su respuesta con las medidas descriptivas adecuadas.

\part[5] Históricamente, la variabilidad porcentual de la glicemia ha sido del 20\%. ¿La muestra obtenida es más o menos homogénea que el registro histórico? Justifique estadísticamente su respuesta.

\part[10] Las personas con glicemia sobre 100, serán parte de un tratamiento experimental. ¿Qué porcentaje de pacientes de este estudio serán parte del tratamiento experimental?

\part[5] ¿Qué porcentaje de los pacientes está bajo el valor de la media?
\end{parts}



\question[30]
Un complejo sistema de ingeniería posee un interruptor automático
de seguridad, que debe activarse en condiciones
de falla del sistema. Suponga que la probabilidad de que
el interruptor se active, dado que hay una falla es $0,99$ y
que la probabilidad de no activarse, dado que no hay falla
también es $0,99$. Finalmente, suponga que la probabilidad
de que el sistema falle es $0,001$. Calcule la probabilidad de
que el sistema haya fallado, sabiendo que el interruptor de
seguridad se activó.


\question
Un asunto de importancia médica es determinar si trotar conduce a una reducción del pulso cardiaco. Para probar esta hipótesis, 8 voluntarios sedentarios acordaron iniciar un programa de entrenamiento en trote por 1 mes. Al finalizar el mes se determinaron los nuevos valores de sus pulsos, y se compararon con los valores que tenían anteriormente. Si los datos son los siguientes, 
\begin{center}
\begin{tabular}{l|cccc cccc | c c}
Sujeto & 1 & 2 & 3 & 4 & 5 & 6 & 7 & 8 & $\overline x$ & $S$\\ \hline
Pulso inicial & 74 & 86 & 98 & 102 & 78 & 84 & 79 & 70 & 83,875 & 11,21781 \\
Pulso final & 69 & 85 & 90 & 105 & 71 & 80 & 69 & 74 & 80,375 & 12,58046
\end{tabular}
\end{center}

Use una significancia $\alpha=5\%$.
\begin{parts}
\part[15] ¿Podemos concluir que trotar ha tenido un efecto reductor sobre el pulso cardíaco?
\begin{solution}
Tal como ha sido planteado el enunciado, la tabla anterior presenta dos muestras pareadas, por lo tanto hacemos el test correspondiente para las diferencias Antes $-$ Después: \pts{1}
\begin{center}
\begin{tabular}{l|cccc cccc | c c}
Sujeto & 1 & 2 & 3 & 4 & 5 & 6 & 7 & 8 & $\overline d$ & $S_d$\\ \hline
diferencia & 5 & 1 & 8 & -3 & 7 & 4 & 10 & -4 & 3,5 &  5,09902.  \\
\end{tabular}
\end{center}
\hfill \pts{2}

Queremos hacer el test:
\begin{align*}
H_0 : \mu_d = 0 \\
H_1 : \mu_d > 0
\end{align*}\hfill \pts{3}

El estadístico a usar es 
\begin{align*}
Z_0 = \frac{\overline d }{S_d / \sqrt{n}} \sim t_{n-1}.
\end{align*}
\hfill \pts{2}

De la tabla $t$ de Student para 7 grados de libertad, hallamos $t_{7 ; 0,95} = 1,895$, por lo rechazamos $H_0$ si $t_{\rm obs} > t_{7 ; 0,95} = 1,895$. \hfill \pts{3}


Calculemos el valor observado del estadístico:
\begin{align*}
t_{\rm obs} = \frac{\overline d }{S_d / \sqrt{n}}  = \frac{3,5 }{ 5,08802 / \sqrt{8}}  = 1,94145.
\end{align*}\hfill \pts{1}


Como $t_{\rm obs} = 1,94145 > t_{7 ; 0,95} = 1,895$, el valor observado  está en la región de rechazo, por lo que se rechaza $H_0$. Concluimos que hay suficiente evidencia estadística para afirmar que hubo un descenso en el pulso cardíaco medio.\hfill \pts{3}

\end{solution}


%\bigskip

\part[15] Imagine que a la tabla anterior le cambiamos ``Pulso inicial'' por ``Sedentarios'', y ``Pulso final'' por ``Trotadores'', de manera que la nueva tabla exprese el pulso de 16 personas distintas, 8 de ellos sedentarios, y 8 de ellos trotadores, sin relación entre ellos. ¿Se puede concluir que los trotadores tienen menor pulso que los sedentarios? \emph{Suponga varianzas iguales.}

\part[5] Comente las similitudes y diferencias de sus conclusiones entre (a) y (b). 
\begin{solution}
Los resultados son dispares entre las dos situaciones, pese a que numéricamente los datos coinciden. La diferencia radica en que en  (a) se asume que los datos representan muestras pareadas, pero en (b) representan muestras independientes. Como los casos representan situaciones  distintas, no hay razón para esperar que las conclusiones coincidan.
\end{solution}

\end{parts}


%%% Diversas preguntas del problema
% \begin{parts}
% 	\part[]
% 	\part[]
% 	\part[]
% \end{parts}

\end{questions}
\end{document}






